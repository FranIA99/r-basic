% Options for packages loaded elsewhere
\PassOptionsToPackage{unicode}{hyperref}
\PassOptionsToPackage{hyphens}{url}
%
\documentclass[
]{article}
\usepackage{lmodern}
\usepackage{amssymb,amsmath}
\usepackage{ifxetex,ifluatex}
\ifnum 0\ifxetex 1\fi\ifluatex 1\fi=0 % if pdftex
  \usepackage[T1]{fontenc}
  \usepackage[utf8]{inputenc}
  \usepackage{textcomp} % provide euro and other symbols
\else % if luatex or xetex
  \usepackage{unicode-math}
  \defaultfontfeatures{Scale=MatchLowercase}
  \defaultfontfeatures[\rmfamily]{Ligatures=TeX,Scale=1}
\fi
% Use upquote if available, for straight quotes in verbatim environments
\IfFileExists{upquote.sty}{\usepackage{upquote}}{}
\IfFileExists{microtype.sty}{% use microtype if available
  \usepackage[]{microtype}
  \UseMicrotypeSet[protrusion]{basicmath} % disable protrusion for tt fonts
}{}
\makeatletter
\@ifundefined{KOMAClassName}{% if non-KOMA class
  \IfFileExists{parskip.sty}{%
    \usepackage{parskip}
  }{% else
    \setlength{\parindent}{0pt}
    \setlength{\parskip}{6pt plus 2pt minus 1pt}}
}{% if KOMA class
  \KOMAoptions{parskip=half}}
\makeatother
\usepackage{xcolor}
\IfFileExists{xurl.sty}{\usepackage{xurl}}{} % add URL line breaks if available
\IfFileExists{bookmark.sty}{\usepackage{bookmark}}{\usepackage{hyperref}}
\hypersetup{
  pdftitle={Documento trabajo practico},
  pdfauthor={Francisco Alberto},
  hidelinks,
  pdfcreator={LaTeX via pandoc}}
\urlstyle{same} % disable monospaced font for URLs
\usepackage[margin=1in]{geometry}
\usepackage{graphicx,grffile}
\makeatletter
\def\maxwidth{\ifdim\Gin@nat@width>\linewidth\linewidth\else\Gin@nat@width\fi}
\def\maxheight{\ifdim\Gin@nat@height>\textheight\textheight\else\Gin@nat@height\fi}
\makeatother
% Scale images if necessary, so that they will not overflow the page
% margins by default, and it is still possible to overwrite the defaults
% using explicit options in \includegraphics[width, height, ...]{}
\setkeys{Gin}{width=\maxwidth,height=\maxheight,keepaspectratio}
% Set default figure placement to htbp
\makeatletter
\def\fps@figure{htbp}
\makeatother
\setlength{\emergencystretch}{3em} % prevent overfull lines
\providecommand{\tightlist}{%
  \setlength{\itemsep}{0pt}\setlength{\parskip}{0pt}}
\setcounter{secnumdepth}{-\maxdimen} % remove section numbering

\title{Documento trabajo practico}
\author{Francisco Alberto}
\date{4/9/2020}

\begin{document}
\maketitle

\#Ejercicios sobre LaTeX, R y Markdown

\#\#\#Juan Gabriel Gomila \& María Santos

\#\#\#\#\#530/12/2018

\#\#\#Instrucciones

En primer lugar, debéis reproducir este documento tal cual está.
Necesitaréis instalar MiKTeX y Texmaker.

A continuación de cada pregunta, tenéis que redactar vuestras respuestas
de manera correcta y argumentada,indicando qué hacéis, por qué, etc. Si
se os pide utilizar instrucciones de R, tendréis que mostrarlas todas en
chunks.

El objetivo de esta tarea es que os familiaricéis con los documentos
Markdown, las fórmulas en \(\LaTeX\) y los chunks de R . Y, de lo más
importante, que os acostumbréis a explicar lo que hacéis en cada
momento.

\#\#\#Preguntas

\#\#\#\#Pregunta 1

Realizad los siguientes productos de matrices siguiente en
R:\[A \cdot\ B\] \[ B \cdot\ A\] \[\ ( A \cdot\ B \ )^t\]
\[\ B^t \cdot\ A\] \[\ ( \ A \cdot\ B \ )^-1\] \[ \ A^-1 \cdot\ B^t\]
donde

Finalmente, escribe haciendo uso de \(\LaTeX\) el resultado de los dos
primeros productos de forma adecuada.

\#\#\#\#Pregunta 2

Considerad en un vector los números de vuestro DNI y llamadlo dni. Por
ejemplo, si vuestro DNI es 54201567K, vuestro vector será,
\[ \ dni \ = \ (5,4,2,0,1,5,6,7 \ )\]

Definid el vector en R. Calculad con R el vector dni al cuadrado, la
raíz cuadrada del vector dni y, por último,la suma de todas las cifras
del vector dni.

Finalmente, escribid todos estos vectores también a \(\LaTeX\)

\#\#\#\#Pregunta 3

\end{document}

% Options for packages loaded elsewhere
\PassOptionsToPackage{unicode}{hyperref}
\PassOptionsToPackage{hyphens}{url}
%
\documentclass[
]{article}
\usepackage{lmodern}
\usepackage{amssymb,amsmath}
\usepackage{ifxetex,ifluatex}
\ifnum 0\ifxetex 1\fi\ifluatex 1\fi=0 % if pdftex
  \usepackage[T1]{fontenc}
  \usepackage[utf8]{inputenc}
  \usepackage{textcomp} % provide euro and other symbols
\else % if luatex or xetex
  \usepackage{unicode-math}
  \defaultfontfeatures{Scale=MatchLowercase}
  \defaultfontfeatures[\rmfamily]{Ligatures=TeX,Scale=1}
\fi
% Use upquote if available, for straight quotes in verbatim environments
\IfFileExists{upquote.sty}{\usepackage{upquote}}{}
\IfFileExists{microtype.sty}{% use microtype if available
  \usepackage[]{microtype}
  \UseMicrotypeSet[protrusion]{basicmath} % disable protrusion for tt fonts
}{}
\makeatletter
\@ifundefined{KOMAClassName}{% if non-KOMA class
  \IfFileExists{parskip.sty}{%
    \usepackage{parskip}
  }{% else
    \setlength{\parindent}{0pt}
    \setlength{\parskip}{6pt plus 2pt minus 1pt}}
}{% if KOMA class
  \KOMAoptions{parskip=half}}
\makeatother
\usepackage{xcolor}
\IfFileExists{xurl.sty}{\usepackage{xurl}}{} % add URL line breaks if available
\IfFileExists{bookmark.sty}{\usepackage{bookmark}}{\usepackage{hyperref}}
\hypersetup{
  pdftitle={Documentacion de Textos},
  pdfauthor={Francisco Alberto},
  hidelinks,
  pdfcreator={LaTeX via pandoc}}
\urlstyle{same} % disable monospaced font for URLs
\usepackage[margin=1in]{geometry}
\usepackage{graphicx,grffile}
\makeatletter
\def\maxwidth{\ifdim\Gin@nat@width>\linewidth\linewidth\else\Gin@nat@width\fi}
\def\maxheight{\ifdim\Gin@nat@height>\textheight\textheight\else\Gin@nat@height\fi}
\makeatother
% Scale images if necessary, so that they will not overflow the page
% margins by default, and it is still possible to overwrite the defaults
% using explicit options in \includegraphics[width, height, ...]{}
\setkeys{Gin}{width=\maxwidth,height=\maxheight,keepaspectratio}
% Set default figure placement to htbp
\makeatletter
\def\fps@figure{htbp}
\makeatother
\usepackage[normalem]{ulem}
% Avoid problems with \sout in headers with hyperref
\pdfstringdefDisableCommands{\renewcommand{\sout}{}}
\setlength{\emergencystretch}{3em} % prevent overfull lines
\providecommand{\tightlist}{%
  \setlength{\itemsep}{0pt}\setlength{\parskip}{0pt}}
\setcounter{secnumdepth}{-\maxdimen} % remove section numbering

\title{Documentacion de Textos}
\author{Francisco Alberto}
\date{11/8/2020}

\begin{document}
\maketitle

\hypertarget{titulo-1}{%
\section{Titulo 1}\label{titulo-1}}

\hypertarget{titulo-2}{%
\subsection{Titulo 2}\label{titulo-2}}

\hypertarget{titulo-3}{%
\subsubsection{Titulo 3}\label{titulo-3}}

\hypertarget{titulo-4}{%
\paragraph{Titulo 4}\label{titulo-4}}

\hypertarget{titulo-5}{%
\subparagraph{Titulo 5}\label{titulo-5}}

Titulo 6

Esto es un texto llano. Podemos escribir sin problemas todo el texto que
queramos para acompañar tanto al código en \texttt{R} como las funciones
en \LaTeX.

Esto seria una nueva linea de texto

Para usar la \emph{cursiva} podemos: escribir con un asterisco:
\emph{Esto es cursiva} , o tambien con un guion bajo: \emph{Esto es
cursiva}

Para usar \textbf{negrita} podemos escribir: con doble asterisco
\textbf{Esto es una negrita} o con dos guiones bajos \textbf{Esto es una
negrita}

Los superindices van con el \textbf{sombrerito}: Mi
nota\textsuperscript{2} es genial.

Para tacharuna palabra usamos doble tilde: \sout{Las matemáticas son un
rollo}.

Visita nuestra empresa, haz click aqui:
\href{https:/www.empresaderobotica.com/}{empresa de robótica}

imagen \includegraphics{../../../../}

\hypertarget{listas-no-ordenadas}{%
\subsubsection{Listas no ordenadas}\label{listas-no-ordenadas}}

\begin{itemize}
\tightlist
\item
  item primero
\item
  item segundo

  \begin{itemize}
  \tightlist
  \item
    subitem 2.1
  \item
    subitem 2.2
  \item
    subitem 2.3
  \end{itemize}
\item
  item tercero
\end{itemize}

\end{document}
